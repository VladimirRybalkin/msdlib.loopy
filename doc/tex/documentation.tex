\documentclass{report}

\usepackage{graphicx}
\usepackage[utf8]{inputenc}
\usepackage[english,german]{babel}
\usepackage[pdftex, pdfborder={0 0 0}]{hyperref}
\usepackage[nameinlink]{cleveref}
%\usepackage{alltt}
%\usepackage{multirow}
%\usepackage{rotating}
\usepackage{listings}
\usepackage{setspace}

%\usepackage{fancyhdr}
%\setlength{\headheight}{15.2pt}

\usepackage[%gray,
      dvipsnames,svgnames]{xcolor}
\definecolor{light-green}{rgb}{0.9,1,0.85}

\usepackage{tikz}
\usetikzlibrary{matrix,arrows,shadows,positioning,fit,calc,decorations.pathreplacing,decorations.pathmorphing,decorations.shapes,decorations.text,automata}
\tikzset{math mode/.style = {execute at begin node=$, execute at end node=$}}

%tikz
\pgfdeclarelayer{background}
\pgfdeclarelayer{foreground}
\pgfsetlayers{background,main,foreground}

% boxes
\tikzstyle{class}=[draw, text width=8em, minimum height=2.5em, text centered]
\tikzstyle{tool}= [draw, text width=8em, minimum height=2.5em, text centered, draw=RoyalBlue, fill=blue!15 ]
\tikzstyle{file} = [draw, text width=8em, minimum height=2.5em, text centered, draw=ForestGreen, fill=light-green, rounded corners]
\tikzstyle{group}=[inner sep=10pt,draw=gray, very thick, dotted, rounded corners=4pt]

% arrows
\tikzstyle{arr}=[->, >=stealth', thick, shorten <=0.2mm, shorten >=0.2mm,gray]
\tikzstyle{aggregate}=[arr, >=diamond]

%others
\tikzstyle{ann} = [above, text width=5em, text centered]

\lstset{literate=%
{Ö}{{\"O}}1
{Ä}{{\"A}}1
{Ü}{{\"U}}1
{ß}{{\ss}}2
{ü}{{\"u}}1
{ä}{{\"a}}1
{ö}{{\"o}}1
{𝜺}{{$\epsilon$}}1
{∧}{{$\wedge$}}1
{∨}{{$\vee$}}1
}

%listing languages
\lstdefinelanguage{xcend}{
  keywords={element, attribute, assert, exists, not exists, size, count, sum, tally, insert, update, delete, if, then, else, fi, at, to, using},
  comment=[l]{//},morecomment=[n]{/*}{*/}
}

\lstset{
   %language=katja,
   basicstyle=\footnotesize,
   backgroundcolor=\color{light-green},
   frame=single,
   framerule=0pt
}

\renewcommand\arraystretch{1.2}
\widowpenalty=300
\clubpenalty=300

\begin{document}

\selectlanguage{english}
\title{HOPP Driver Generator Documentation}
\author{Thomas Fischer}
\date{\today}

\maketitle
\thispagestyle{empty}
\newpage

\tableofcontents
\thispagestyle{empty}
\newpage

\chapter{Introduction}
This document is intended to give an overview over the HOPP Driver Generator for both, users and developers. Users can find an easy to understand explanation of the generator as well as usage examples in \Cref{sec:userDoc}. Developers and other interested readers can find details about the architecture and implementation of the generator in \Cref{sec:devDoc}.
Here be an introduction to the system \cite{fischer12}.

%define scope of the generator
The HOPP Driver Generator is intended to provide embedded system developers %that's kinda vague
with a simple, convenient interface to communicate with their designed hardware components. % that's not better...

\chapter{User-Documentation}
\label{sec:userDoc}


\section{Command Line Interface}
The HOPP Driver Generator can be called using the command line interface (CLI). While the tool generates C/C++ code, it is written in Java and therefore requires an installation of Java 6 or above. The CLI offers several parameters to futher configure the run of the generator.

\begin{table}
\centering
\begin{tabular}{ cc | p{9cm} }
-d & --dest & Specifies the destination directory. All files will be generated into the specified directory. If none is specified, the current working directory will be used instead.\\ \hline
& --debug & The driver will be generated with additional debug console output. This makes it easier to track errors in larger test cases.\\ \hline
& --mac & used to set the mac address of a possible Ethernet interface of the board. Notation: XX:XX:XX:XX:XX:XX, where each X marks a hexadecimal number (allowing lower as well as upper cases).\\
& --ip & used to set the ip address of a possible Ethernet interface of the board. Notation: X.X.X.X, where each X marks a decimal number ranging from 0 to 255. The same notation is required for the following two parameters.\\
& --mask & used to set the network mask of a possible Ethernet interface of the board.\\
& --gw & used to set the standard gateway of a possible Ethernet interface of the board. \color{red}{Note that these four parameters are only contemporary and will be replaced by the new board description language (I hope).}\\ \hline
-h & --help & Lists all CLI parameters and a short explanation. The generator will abort after parsing this parameter and not generate anything.
\end{tabular}
\end{table}

Additionally, the \texttt{.mhs} file of the system is required by the generator. So a complete call looks like \texttt{java -jar driverGenerator.jar [OPTIONS] <.mhs file>}.

\section{Resulting Artifacts}
The output of the generator consists of two groups of files. The first group contains all files that make up the board side of the driver, which have to be compiled to an \texttt{.elf} file by the Xilinx SDK. The second group are files for the client side, that can be used to wrap communication with the board and its components.

\subsection{Board Part}
For the board, only two relevant files are generated, a header and a plain c source file. Both files have to be imported in a new Xilinx SDK project started from an existing XPS project. The \texttt{.elf} file generated from this project then has to be uploaded to the target board.

\subsection{Client Part}
The client side currently consists only of a header and a plain c source file as well. These files can be imported in any client side project and be used to wrap communication. Currently, it is only possible to set the LEDs of the board.

\section{API}
%why not do all this stuff statically? Can there be more than one? When the driver has been generated for this specific board?? Or should I do the configuration more dynamically??
%Question: more general classes and then instantiate at runtime? Or should I provide concrete classes with fixed methods extending the general classes? 
%While the first one would probably be more flexible and also allow relatively easy modifications of the driver without rebuilding, the latter provides more type safety, since the more generic methods taking parameters (like the internal port identifier or memory address to write to) can be completely hidden from the user. I prefer the latter one... (which would probably again underline possibility of a more model...

\subsection{Component}
For each port on the component, a method returning the corresponding port object will be generated.

\subsection{Port}
A port object contains methods for receiving or sending data, depending on the type of the port. Naturally, receiving ports only contain methods for receiving data and sending ports only for sending. Dual ports contain the methods of both. Ports are restricted to a specific data type, i.e. bitwidth.

\chapter{Developer-Documentation}
\label{sec:devDoc}
This section is addressed to future developers of the Driver Generator and describes the architecture and implementation of the generator itself.

\section{Overview}

\subsection{JFlex \& Cup}
This can be kept short with a reference to the JFlex/Cup documentation

\subsection{Katja}
The driver generator uses the Katja tool, developed by the Software Technology Group of the University of Kaiserslautern. This tool generates several data types\footnote{Since these data types will be described using their Katja specifications, it is strongly recommended to read through the Katja specification provided in form of three technical reports at \url{https://softech.informatik.uni-kl.de/Homepage/Katja}}. To be more specific, it provides the AST build up by the CUP parser as well as a model of the C/C++ language described in detail in \Cref{sec:cmodel}.

\section{Architecture}
The overall idea of the driver generator is described by \Cref{fig:process}. The source \texttt{.mhs} file describing the system board design is first translated into an internal representation of the board, i.e. an abstract syntax tree. This AST is used as input for the generator, which in turn outputs all required header- and source files.

\begin{figure}
\centering
\begin{tikzpicture}[scale=1, transform shape]
\coordinate(center);
\node[tool] (gen) at (0,0) {Generator};

\node[file, above=1cm] (ast) at (gen) {Board AST};
\node[tool, above=1cm] (parser) at (ast) {Parser};
\node[file, above=1cm] (mhs) at (parser) {.mhs file};

%make filegroups out of these... yes, it's not trivial ;)
\node[file, below left = 1.5cm and 0cm of gen] (boardh) {Board.h};
\node[file, below right = 1.5cm and 0cm of gen] (hosth) {Host.h};

\node[file, below=1cm] (boardc) at (boardh) {Board.c};
\node[file, below=1cm] (hostc) at (hosth) {Host.c};

\node[group, fit=(boardh) (boardc), label={95:board side}] (board) {};
\node[group, fit=(hosth) (hostc), label={75:host side}] (host) {};

\draw[arr] (mhs) to node[auto] {input} (parser);
\draw[arr] (parser) to node[auto] {generates} (ast);
\draw[arr] (ast) to node[auto] {input} (gen);
\draw[arr] (gen) to node[left] {generates} (board);
\draw[arr] (gen) to node[right] {generates} (host);

\draw[arr, <->] (board) to node[below] {communicates} (host);
\end{tikzpicture}
\caption{A rough sketch of the translation process so far}
\label{fig:process}
\end{figure}

\begin{figure}
\centering
\begin{tikzpicture}[scale=1, transform shape]
\coordinate(center);
\node[class] (component) at (0,0) {Component};
\node[class, below left=2.5cm and 1.5cm] (interface) at (component) {Interface};
\node[class, below right=2.5cm and 1.5cm] (port) at (component) {Port};

\node[above=8em] (anchor1) at (interface) {};
\node[above=8em] (anchor2) at (port) {};

\draw[arr, -, shorten >= -0.4em] (interface) to node[auto] {} (anchor1);
\draw[arr, -, shorten >= -0.4em] (port) to node[auto] {} (anchor2);

\draw[aggregate, shorten <= -0.35em] (anchor1) to node[auto] {} (component);
\draw[aggregate, shorten <= -0.35em] (anchor2) to node[auto] {} (component);

\node[ann, above left = 0em and 3.25em] (arr1) at (component) {1};
\node[ann, above right = 0em and 3.25em] (arr2) at (component) {1};

\node[ann, above right= 1.5em and -2em] (arr3) at (interface) {1};
\node[ann, above right = 1.5em and -1.5em] (arr4) at (port) {1..n};

\end{tikzpicture}
\caption{Architecture of the host part (to come)}
\label{fig:archHost}
\end{figure}

The (intended) architecture of the host software is depicted in \Cref{fig:archHost}. The core concept of the host software is the \textit{component}. Such a component is a designed hardware unit, which can have several \textit{ports}, over which data can be sent to or received from the component. Hidden from the user, a component also has exactly one \textit{interface}, which handles actual communication.

\subsection{Component}
A component is a piece of designed hardware. It has multiple ports over which communication can take place, i.e. date is sent from or to the component, esp. the control ports \textit{clock} and \textit{reset}. Usually components receive data, process it and send back some results, though possibly on another port.

These components are described using VHDL. Such VHDL definitions can then be referenced in the board description language provided by the HOPP Driver Generator. The VHDL files are parsed and translated into components with corresponding ports as defined within the files.

\subsection{Port}
Ports are used to send data to or receive data from components. A port is always assigned to a single component, but a component can have multiple ports. Ports can be receiving ports, sending ports or dual ports. They can have an arbitrary bitwidth. Since the bitwidth of the microblaze as well as the arm processor is limited to 32-bit, the port class has to split greater bitwidths in several 32-bit packages, but hide this process from the user. The encoding format, i.e. order of bits, also is hidden from the user.

\subsection{Communication Interface}
The communication interface is hidden from the user. Each component has exactly one such interface and a communication interface has to be attached to a component. The communication interface wraps all communication between host software and board software. The interface abstracts from the physical interface and provides a homogeneous api for components and ports. Network interface specific initialization is generated as well and is not required by the user (other than annotating configuration details in the board description).

\section{Generation Backend}
This section is used to introduce the C generation backend. A reference to the third Katja report might be useful.

\subsection{CModel}
\label{sec:cmodel}
The c model provides data types representing a c/c++ program. Note, that the model is neither complete nor always valid, i.e. not all c programs can be described using this model and it is possible to specify a model not translating into valid c. Still, the model simplifies the process of code generation. The model is used for generating C as well as C++ code.

\begin{lstlisting}[language=java, breaklines=true]
MFile ( String name, MDefinitions defs, MStructs structs, MEnums enums, MAttributes attributes, MMethods methods, MClasses classes )
MClass ( MModifiers modifiers, String name, MTypes extend, 
	 MStructs structs, MEnums enums, MAttributes attributes, MMethods methods, MClasses nested )
MModifier = PRIVATE() | PUBLIC() | CONSTANT() | STATIC() | INLINE()
\end{lstlisting}

First of all, each file has a name. A file consists of several definitions, structures, enums, attributes and methods. Files also can contain several classes. These classes again have a name and can contain all these components including other classes. In addition, classes can contain modifiers and inherit components from other classes.
The allowed modifiers are private, public, constant, static, and inline. Note, that not all of these modifiers are class modifiers, and several combinations of modifiers are invalid (e.g. private and public). The model relies on the developer to choose modifiers according to the modified program part.

\begin{lstlisting}[language=java, breaklines=true]
MDefinition ( String name, String value)
MStruct ( MModifiers modifiers, String name, MAttributes attributes )
MEnum   ( MModifiers modifiers, String name, Strings values )
\end{lstlisting}

Definitions, structs and enums mark rather trivial tuple productions. A definition simply assigns a name to a value. %something about structs
Enums list a number of possible values.

\begin{lstlisting}[language=java, breaklines=true]
MAttribute ( MModifiers modifiers, MAnyType type, String name, MCodeFragment initial )
MCodeFragment ( String part, MIncludes needed )
\end{lstlisting}

Attributes are similar to definitions, but are typed and may also be left unassigned, using an empty code fragment. The \texttt{MIncludes} is required if the type of the attribute is not defined within this c file itself.

\begin{lstlisting}[language=java, breaklines=true]
MMethod ( MModifiers modifiers, MReturnType returnType, String name, MParameters parameter, MCode body )
MParameter ( MParamType refType, MAnyType type, String name )
MReturnType = MAnyType | MVoid()     | MNone()
MParamType  = VALUE()  | REFERENCE() | CONSTREF()
MCode ( Strings lines, MIncludes needed )
\end{lstlisting}

Methods have a return type and a list of parameters. The method body is also more complex than a simple code fragment and can consist of several lines, which are not checked any further in this model. The return type can be any c type as well as void. For constructors in c++, the return type \texttt{MNone} is used. Parameters also have a type and name. Furthermore, the mode of parameter passing has to be specified.

\begin{lstlisting}[language=java, breaklines=true]
MAnyType = MType             ( String name )
         | MArrayType        ( MAnyType type, Integer length )
         | MPointerType      ( MAnyType type )
         | MConstPointerType ( MAnyType type )
\end{lstlisting}

The type system of the model supports arrays as well as (const) pointers. The basic type is the \texttt{MType}, which consists only of a string that has to reference an existing c type, e.g. "int" or "struct student". This type can then be extended using pointer or array types. So \texttt{MarrayType(MType("int"), 5)} would mark an integer array of length 5. Note, that these types are nested semantically rather than in the same order as in C. Consequently, a point type of a const pointer type of type integer will be translated into \texttt{int const ** a}.

\subsection{Unparser}
Different unparsers are used, to translate the C model in actual code. For each instance of the model, a header file and a corresponding source file, either C or C++, has to be generated. These unparsers are called depending on the particular instance of the model. Files intended to be loaded to the board have to be plain C, files intended for the client side can also be C++ files. The unparsing itself is realized using the visitor pattern. Each visit method appends code to a string buffer depending on the visited element.

\paragraph{Header Unparser}
The header unparser is used for both, unparsing C as well as C++ code. Consequently, it doesn't filter any constructs, but accepts everything specifiable with the model.

This unparser generates only signatures for all methods and only the declarations of attributes and enums. However, the header file will contain all includes referenced within the model.

\paragraph{Plain C Unparser}
Since plain C doesn't have any concept of classes, using a model with classes in this unparser will result in exceptions. Otherwise, all components and combinations are accepted. The source file will only import the corresponding header file, no other headers or sourcefiles.

\paragraph{C++ Unparser}
For the C++ unparser, like with the header unparser, every component and combination is allowed. The source file will only import the corresponding header file, no other headers or sourcefiles.

\section{Transformation}
Explain the transformation of the source file into the c model. Maybe also explain the generated code??

\addtocontents{toc}{\protect\vspace*{\fill}}

\newpage
\pagestyle{empty}
\bibliography{bib}
\bibliographystyle{splncs03}
\end{document}